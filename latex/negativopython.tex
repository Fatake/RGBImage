\documentclass[letterpaper,12pt]{article}
\usepackage{tabularx}
\usepackage{amsmath}
\usepackage{graphicx}
\usepackage[margin=1in,letterpaper]{geometry}
\usepackage{cite}
\usepackage[final]{hyperref} 
\hypersetup{
	colorlinks=true,
	linkcolor=blue,
	citecolor=blue,
	filecolor=magenta,
	urlcolor=blue         
}
\usepackage{blindtext}


\begin{document}

\title{Análisis de imágenes en espacios de color RGB}
\author{Ruiz Lozano Paulo Cesar}
\date{\today}
\maketitle

\begin{abstract}
In this experiment we studied pendulum motion in a non-uniformly accelerating reference frame. \textbf{Special notes:} Prof Goldman specifically requested to not have ``100" sig figs, put font on figure axes in readable size, and don't put grids on plots!
\end{abstract}

%introduccion
\section{Introducción}

En esta tarea se realiza el análisis de la imagen de un gato, por espacios de colores y se expone el algoritmo aplicado en 4 diferentes lenguajes de programación para generar el negativo de una imagen.
Los 4 lenguajes de programación utilizados son:
\begin{itemize}
	\item Python
	\item Java
	\item C
	\item Ruby
\end{itemize}

%Marco teorico
\section{Marco teórico}

Sea \textit{I} el conjunto de matrices en los espacios de color de una imagen, 
donde cada \textit{i} $\in$  \textit{I} es de la forma ( \textit{u}, \textit{u} , \textit{u} ), \textit{u} $\in$ ( 0, 255 )

\subsection{Negativo de una imagen}
El negativo \textit{$\ddot{V}$} de una imagen se define como:
\begin{equation}
   \ddot{V} = 255 - u
   \label{Eq:equation1} %the label lets you refer to the equation later
\end{equation}
Entonces, cada imagen \textit{I} tiene 3 espacios de color \textbf{RGB}.
Para obtener el negativo se usa el siguiente algoritmo:
\begin{gather*}
for (\textit{i} \in  \textit{I}) do:
	i[a,b] \Longleftarrow 255 - i[a,b]
\end{gather*}



\section{Python}
Código en python:
\begin{minted}{python}
  Se divides la imagen por canales
  chanel_r = imagen[:,:,2]//Canal R
  chanel_g = imagen[:,:,1]//Canal G
  chanel_b = imagen[:,:,0]//Canal B

  Se obtiene las dimensiones
  row, col = chanel_r.shape

  Se generan los canales negativos
  neg_r = np.zeros((row, col), dtype=np.uint8)
  neg_g = np.zeros((row, col), dtype=np.uint8)
  neg_b = np.zeros((row, col), dtype=np.uint8)

  Se obtiene el negativo de cada Canal
  for i in range(0, row):                                            
      for j in range(0, col):                                          
          neg_r[a, b] = 255 - chanel_r[a, b]

  for i in range(0, row):                                            
      for j in range(0, col):                                          
          neg_g[a, b] = 255 - chanel_g[a, b]

  for i in range(0, row):                                            
      for j in range(0, col):                                          
          neg_b[a, b] = 255 - chanel_b[a, b]

  negativo = imagen
  negativo[:,:,2] = neg_r
  negativo[:,:,1] = neg_g
  negativo[:,:,0] = neg_b
\end{minted}

%Conclusion
\section{Conclusions}
Conclusiones del trabajo


\begin{thebibliography}{99}

\bibitem{melissinos}
A.~C. Melissinos and J. Napolitano, \textit{Experiments in Modern Physics},
(Academic Press, New York, 2003).

\bibitem{Cyr}
N.\ Cyr, M.\ T$\hat{e}$tu, and M.\ Breton,
% "All-optical microwave frequency standard: a proposal,"
IEEE Trans.\ Instrum.\ Meas.\ \textbf{42}, 640 (1993).

\bibitem{Wiki} \emph{Expected value},  available at
\texttt{http://en.wikipedia.org/wiki/Expected\_value}.

\end{thebibliography}


\end{document}
